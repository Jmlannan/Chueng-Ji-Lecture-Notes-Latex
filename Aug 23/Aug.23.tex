\documentclass{article}
\usepackage{blindtext}
\usepackage[utf8]{inputenc}
\usepackage{braket}
\usepackage{amsmath}
\usepackage{commath}
\usepackage{physics}
\title{August 23}
\author{Joseph Lannan}
\date{\today}
\begin{document}
\section{Bras, Kets, and Operations}
Continuing our analogy to electromagnetic waves, we can write these waves as vectors such
$$
\vec{E} = 
		\begin{cases}
			E_0 \hat{x} \cos(kz - \omega t) = R_e [E_0 \hat{x}je^(i(kz - \omega t) \ket{+} \\
			E_0 \hat{x} \cos(kz - \omega t) = R_e [E_0 \hat{y}je^(i(kz - \omega t) \ket{-}
	   \end{cases} \\
$$ \\
$$
Re E_0 \frac{ \hat{x} + \hat{y}}{ \sqrt[2]{z}} e^(-i(kz - \omega t)) = \ket{ S_x^+ } = \frac{ \ket{+} + \ket{-}}{ \sqrt[2]{2}} $$ \\ 
$$
Re  E_0 \frac{- \hat{x} + \hat{y}}{ \sqrt[2]{z}} e^(i(kz - \omega t)) = \ket{ S_x^- } = \frac{- \ket{+} + \ket{-}}{ \sqrt[2]{2}}
$$
Circularly polarized wave
$$
Re E_0 \frac{ \hat{x} + \iota \hat{y} }{ \sqrt{2}} e^( \iota (kz - \omega t)) = \ket{S_y^+} = \frac{ \ket{+} + \iota \ket{-}}{ \sqrt{2}}$$ \\
$$
Re E_0 \frac{ \hat{x} + \iota \hat{y} }{ \sqrt{2}} e^( \iota (kz -\omega t)) = \ket{S_y^+} = \frac{ \ket{+} + i \ket{-}}{ \sqrt{2}}
$$

Complex vector space must be introduced to cover the quantum space
Dimension of the complex vector space depends on the system. 
e.g. $ Ag^47 $ : spin-$ \frac{1}{2} $ system, dimension 2
spin-s system, $dim = 2s+1 $
Hilbert space- Nondenumerably infinite dimensional complex vectors pace (Quantum Mechanical Space)
ef. Euclidean Space
denumerable (countable); N(natural numbers
nondenumerable (uncountable); real numbers, including irrational numbers
While we usually indicate a finite number of dimensions N, of the ket space, the results can immediately be generalized to nondenumerably infinite dimensions. In quantum mechanics, a physical state (e.g. $Ag^47$ atom) with a finite spin orientation is represented by a state vector in a complex vector space. Following P.A.M Dirac's notations, we call such a state a ket and denote it by $\ket{ \alpha }$. This state ket is postulated to contain complete information about the physical state, i.e. everything that we are allowed to ask about the state is contained in the ket.
An observable can be represented by an operation and denoted by A in the vector space in and question. Quite generally, an operator acts on a ket from the left:
$$
A(\ket{\alpha}) = A\ket{\alpha)}
$$
$A\ket{\alpha)}$ is yet  ket. In general, $A\ket{\alpha}$ is not a constant times $\ket{\alpha}$. However, there are particular kets of importance, known as eigenkets of operator A denoted by $\ket{a'}, \ket{a""},...$. They have the property
$$
A\ket{a'} = a'\ket{a}, A\ket{a''} = a''\ket{a''},...
$$
where $a',a'',..$ are just numbers
e.g.
$$
S_z\ket{S_z^+} = \frac{\bar{h}}{2}\ket{S_z^+}
$$ \\ $$
S_z\ket{S_z^-} = -\frac{\bar{h}}{2}\ket{S_z^-}
$$
$c\ket{\alpha}$ and $\ket{\alpha}$ and represent the same physical state as long as $c \neq 0$
Any arbitrary ket $\ket{\alpha}$ can be written as $\ket{\alpha} = \sum_{a'} C_{a'}\ket{a'}$ with $a', a'',..$ where $C_a'$ is a complex coefficent
e.g. $$
\ket{S_y^(\pm)} = \frac{1}{\sqrt{2}}\ket{S_z^+} \pm \frac{ \iota }{ \sqrt{2} } \ket{ S_z^- }
$$
At this point, it is convenient to introduce a dual space the Bra space to the ket space. In this space complex numbers become their conjugate
$$
c \leftrightarrow c^* 
$$ \\ $$
\bra{ \alpha } \leftrightarrow \ket{ \alpha }
$$ \\ $$
c \bra{ \alpha } \leftrightarrow c^* \ket{ \alpha }
$$
e.g. $\ket{S_y^(\pm)} = \frac{1}{ \sqrt{2} } \ket{S_z^+} \mp \frac{ \iota} { \sqrt{2} } \ket{S_z^-}$
section{Inner Product}
$\braket{\beta}{\alpha} = \braket{\alpha}{\beta}^* \neq \braket{\alpha}{\beta}$ \\
e.g. $ \braket{S-y^-}{S_z^-}  = \frac{1}{\sqrt{2}}  = \braket{S_z^-}{S_y^-}^* = - \frac{\iota}{\sqrt{2}}^*$
 The inner product corresponds to the dot product of the euclidean space 
 $ \braket{\alpha}{\alpha}$ is real because $\braket{\alpha}{\alpha} = \braket{\alpha}{\alpha}$
 $\braket{\alpha}{\alpha} = 0$ if $ket{\alpha}$ is a null ket
 From a physicist's point of view, this point of view, this postulate is essential for the probabilistic interpretation of and quantum mechanics. A normalized ket,
 $\ket{\tilde{\alpha}} = \frac{1}{ \sqrt{ \braket{\alpha}{\alpha} }} \ket{ \alpha }$, since $ \braket{ \tilde{\alpha}}{ \tilde{\alpha}} = 1$
 $\sqrt{\alpha	}$is the norm of $\ket{\alpha}$ analogous to the norm in and euclidean geometry
 Since $\ket{\alpha}$ and $c \ket{\alpha}$ represent the same physical state, we may require that kets be normalized in the sense of $\braket{\alpha}{\alpha} = 1$
 If $\braket{\alpha}{\beta} = 0$ then $\ket{\alpha}$ and $\ket{\beta}$ are said to be orthogonal to each other
 e.g. $\braket{S_z^+}{S_z^-} = 0$
 Now, from $\ket{\alpha} = \sum _(a') \ket{a'}\braket{a'}{\alpha}$
 Closure or Completeness Relation
 $ \braket{ \alpha}{ \alpha} = \sum_{a'} \abs{ \braket{a'}{ \alpha}}^2 = 1$
 the sum of the  and probability is unity
\section{Properties of Operations}
1. Two operators are the same (or equal) X = Y if $ X\ket{\alpha} = Y\ket{\alpha}$ for any $\ket{\alpha}$
2. X is a null operator if $X\ket{\alpha} = 0$ for any $\ket{\alpha}$
3. Addition of operators is commutative $ X + Y = Y + X$
4. Addition of operators is associative $ X + ( Y + X) = (X + Y) + X$
5. and Multiplication of operators is not and necessarily communicative (compatible) $XY \neq YX$
The adjoin of an operator
$X \leftrightarrow X^\dagger$ if $X = X^\dagger$ then $X$ is said to be Hermitian or Self-Adjoint
Theory: The eigenvalues of a Hermititan operator A are real; the eigenkets of A and correspond to different eigenvalues are orthogonal.
Proof: 
$$A \ket{a'} = a' \ket{a'}, \bra{a'}\dagger{A} = a'^*\ket{a'}
$$ \\ $$
\ket{a'}A\bra{a'} = a' = a^*
$$\\
because $\braket{a'}{a'} = 1$\\
$A \ket{a'} = a' \ket{a'} and A\ket{a''} = a'' \ket{a''}$\
then $\bra{a''}A\ket{a} = a' \braket{a''}{a'}$\\
and $\bra{a''} \dagger{A} \ket{a} = a''^* \braket{a''}{a'}$\\
Hemiticity implies $\bra{a''} A \ket{a'} = a'' \braket{a''}{a'}$\\
$0 = (a' - a'') \braket{a''}{a'}$ \\
Since $a' \neq a'', \braket{a''}{a'} = 0$ or $\ket{a'}$ and $\ket{a''}$ are orthogonal
\section{Observables as an Outerproduct of Bras and Kets}
$A = \sum_(i=1)^N a^i \ket{a^i} \bra{a^i}$\\
because $A \ket{a^i} = a^i \ket{a^i} \bra{a^i}$
so $A \ket{a^i} \bra{a^i} = a^i \bra{a^i} \ket{a^i}$ \\
and $A \sum_(i=1)^N \ket{a^i} \bra{a^i} = \sum_(i=1)^N a^i \ket{a^i} \bra{a^i}$ \\
$\sum_(i=1)^N \ket{a^i} \bra{a^i} = I the identity matix$ \\
therefore $A = \sum_(i=1)^N a^i \ket{a^i} \bra{a^i}$ \\
e.g. $S_z = \frac{\bar{h}}{2} \ket{S_z^+}\bra{S_z^+} - \frac{\bar{h}}{2}$ \\ $\ket{S_z^-}\bra{S_z^-}$
 $= \frac{\bar{h}}{2	}(\ket{+}\bra{-} - \ket{-}\bra{-})$
\end{document}