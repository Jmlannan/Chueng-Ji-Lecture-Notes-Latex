\documentclass{book}
\usepackage{physics}
\usepackage{amsmath}
\begin{document}
\chapter{Matrix Representations}
In the last lecture, we introduced kets, bras and operators. In this Chapter, we will show that they can be represented by column vectors, row vectors and matrices respectively. This is possible due to the closure on completeness relation. 
i.e. $I = \sum^N_{i=1} \ket{a^(i)} \bra{a^(i)} (i=1,2,...,N)$
Identify operator is the sum of each and every orthonormal eigenket and corresponding eigenbra, where orthonormal means
$$ \braket{a^(i)}{a^(i)} = \delta_{ij}$$
where $\delta$ is the dirac delta function such that
$$ \delta_{ij} = 
\begin{cases}
1, i = j \\
0, i \neq j
\end{cases}
$$
e.g. In spin-$\frac{1}{2}$ systems,
$$ I = \ket{+}\bra{+} + \ket{-}\bra{-}$$
where 
$$ \braket{+}{+} = 1, \braket{+}{-} = \braket{-}{+} = 0, \braket{-}{-} = 1$$
these results can be obtained by representing the vectors as
$$\ket{+} = \begin{bmatrix} 1 \\ 0 \end{bmatrix}$$ 
$$\ket{-} = \begin{bmatrix} 0 \\ 1 \end{bmatrix} $$
corresponds to
$$ \bra{+} = \begin{bmatrix} 1 & 0 \end{bmatrix} $$
$$\bra{-} = \begin{bmatrix} 0 & 1 \end{bmatrix} $$

\section{Inner Products of Bras and Kets}
$$ \braket{+}{+} = \begin{bmatrix} 1 & 0 \end{bmatrix} \begin{bmatrix} 1 \\ 0 \end{bmatrix} = 1 x 1 + 0 x 0 = 1 $$
$$ \braket{+}{-} = \begin{bmatrix} 1 & 0 \end{bmatrix} \begin{bmatrix} 0 \\ 1 \end{bmatrix} = 1 x 1 + 0 x 0 = 1 $$
\section{Outer Products of Bras and Kets}
$$\ket{+} \bra{+} = \begin{bmatrix} 1 \\ 0 \end{bmatrix}  \begin{bmatrix} 1 & 0 \end{bmatrix} = \begin{bmatrix} 1 & 0 \\ 0 & 0 \end{bmatrix}$$
$$\ket{+} \bra{-} = \begin{bmatrix} 0 \\ 1 \end{bmatrix}  \begin{bmatrix} 1 & 0 \end{bmatrix} = \begin{bmatrix} 0 & 0 \\ 1 & 0 \end{bmatrix}$$
$$ I = \ket{+} \bra{+} + \ket{+} \bra{-} = \begin{bmatrix} 1 & 0 \\ 0 & 0 \end{bmatrix} + \begin{bmatrix} 0 & 0 \\ 1 & 0 \end{bmatrix} $$
We can extrapolate this to an N-dimensional system with column vectors $\ket{a_1}$, $\ket{a_2}$ ... $\ket{a_N}$
$$\ket{a_1} = \begin{bmatrix} 1 \\ 0 \\ \vdots \\ 0 \end{bmatrix}$$
$$\ket{a_2} = \begin{bmatrix} 0 \\ 1 \\ 0 \\ \vdots \\ 0 \end{bmatrix}$$
and so on for up to $\ket{a_N}$ where each column vector has N entries. The corresponding bras or row vectors are
$$\bra{a_1} = \begin{bmatrix} 1, & 0, & \hdots & 0 \end{bmatrix}$$
$$\bra{a_2} = \begin{bmatrix} 0, & 1, & 0, & \hdots & 0 \end{bmatrix}$$
These vectors form a complete N-dimensional vector space and we can write any other vector as a sum of coefficients 
$$ \ket{\alpha} = \sum^N C_N\ket{a_N} = \sum^N \ket{a_N} \bra{a_N} \ket{\alpha} $$
\end{document}
