

\documentclass{article}
\usepackage{blindtext}
\usepackage[utf8]{inputenc}
 
\title{Stern-Gerlach August 18}
\author{Joseph Lannan}
\date{\today}
 
\begin{document}
 
\maketitle
 
\section{Introduction}
 
The Stern-Gerlach experiment was first devised by Otto Stern and Walther Gerlach. This experiment results in an observable phenomena that can not be explained classically.


 
\section{The Experiment}
 
In line with the original experiment, we will hypothetically take silver atoms and pass them through a gradient magnetic field of strength $B(x)$. Silver atoms have one unpaired electron (check this with the periodic table if you like) so the atoms will have a magnetic moment
\[\mu = \mu_b\]
where $\mu_b$ is the Bohr magneton. It then follows
\[\vec{V} = -\vec{\mu}\cdot\vec{B}\]
\[\vec{F} = -\nabla\vec{V}\]
since $\mu$ is a constant, we can say
\[\vec{F} = -\vec{\mu}\cdot\nabla\vec{B}\]
For simplicity, we will say that the magnetic field only changes in the z direction and at a constant rate $b$ 
\footnote{It isn't actually possible to create such a field, but as we will see later, the procession of the x and y magnetic moments shall cancel out much of the x and y force}.
The force then becomes
\[\vec{F} = -\vec{\mu}\cdot b\hat{z}\]
If we interpret this classically, the direction of $\mu$ is arbitrary, it can point in any direction, we will expect to see the atoms coming out of the experiment to be smeared out.

This is not however what we see from the Stern-Gerlach experiment.

\end{document}

